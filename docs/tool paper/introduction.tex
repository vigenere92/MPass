\section{Introduction}

\MPass\ is an open source and  efficient tool for the verification of message-passing programs 
communicating in an asynchronous manner over unbounded  channels.
 The verification problem  such as the reachability problem are either undecidable  \cite{BZ83} or have high complexity  \cite{AB93,Rack78,phs-IPL2002,lipton}, even under the assumption that each process is finite-state. 


%
%The verification problem such as reachability problem is undecidable for programs having perfect channels 
%even if the number of states in each process are finite but it becomes decidable if we have 
%lossy, stuttering or unordered channels \cite{AB93}.
%However, to decide the reachability problem in later case we have high complexity obstacles.
%Thus in order to avoid this, one useful approach, \emph{context bounding} \cite{SJ05}, was proposed recently.
%This idea limits the number of context switches between two processes because of which 
%we have a trade off between the extent of verification and computational complexity.\\
\MPass\ verifies the reachability problem of message-passing programs or protocols in which each process is restricted to performing at most $k$ phases (for some natural number $k$).
A {\it phase}  is a run where the process performs  either send or receive operations
(but not both). The bounded-phase  restriction does not limit the \emph{number} of
sends or receives,  and in particular it does not put any restriction on the length of the
run or the size of the buffers.

%
%
%computation with the phase bounded by some natural number. In each phase, a process can perform 
%either send transitions or receive transitions (but not both). The transition consisting of no operation, 
%but just the change of states, can be performed in either of these phases.\\\\

\MPass\ can handle different variants of channel semantics, such as
{\it lossy}, {\it stuttering}, and {\it multiset} that
allow the messages inside the channels to be lost,
duplicated, and re-ordered respectively.



%
\MPass\ is based on the framework   described in \cite{AAC13} which translates bounded phase reachability problem  into the satisfiability of quantifier-free Presburger formula in polynomial time, along with several optimizations. This allows  leveraging the full power of state-of-the-art
{\sc smt}-solvers for obtaining a very efficient solution to the bounded-phase reachability problem for all above mentioned semantics. 


\MPass\ turns this verification task into a push-button exercise,   it can also  be  applied on a number of message-passing program with promising results. More precisely:

%Thus, we can use one of the various SMT solvers which, efficiently, comment on the satisfiabilty of the 
%generated quantifier-free Presburger formula for all the above mentioned semantics.
%It is created in such a way that it can be applied on a number of message paasing program with promising results.


\begin{enumerate}
\item[$\bullet$] \MPass\ is an open source \cite{github.MPass}
  verification tool that analyzes reachability problem in message-passing programs having their phase bounded
\item[$\bullet$] If \MPass\ finds a bug then it is a real bug of the input program.

\item[$\bullet$] If the bound on the number of phases increases  then the set of explored traces increases, and in the limit every run of the programs is then explored.   
\item[$\bullet$] It converts bounded phase reachability problem into quantifier-free presburger formula and then call 
  SMT Solver to output the satisfiability results.
  
%\item[$\bullet$] it performs reachability analysis for faulty protocols also which differs from therir original 
%  one at some or many transition.
\item[$\bullet$] It provides user with the computational time used both by the constraint generation as well as 
  SMT solver to analyze the complexity of the provided protocol. Details regarding these two extractions are given in \cref{sec:logic}.
\item[$\bullet$] It provides user with the satisfiability results in an user-friendly manner by generating a separate tex file.
  More information regarding this can be found within the tool \cite{github.MPass}.
\end{enumerate}

\subsubsection{Targeted users}
\MPass\ can be used by these group of users :
%
\begin{enumerate}
\item Various researchers and Computer scientists can use the freely available and open source 
  code of \MPass\ to compare with other approaches for the verification of message-passing programs, to improve and optimize the implemented techniques (by interfacing with other  other efficient SMT-solver) or by improving the used data structures for handling the protocols, or to target new platforms and programs (e.g., adding shared variables or  other channel semantics).
  
  
  
  
%  optimize it further or improve the 
%  implementation techniques thereby extending its features further   It can also be used
%  
  
\item Teachers of distributed  systems  and concurrent programming classes can use (and augment) \MPass\ 
  to familiarize their students with message-passing programs.  In particular the precision of \MPass\ can concretely illustrate the difficulty of writing a correct concurrent programs.
  
%  
%  They can even illustrate the effects of protocal 
%  and reachability settings provided on the running time of the tool.
\item Software developers who are working on message-passing programs can use \MPass\   to verify their tentative soluteions.
  It can also be used to check faultiness of a particular protocol by verifying the reachability of an Invalid state.
\end{enumerate}