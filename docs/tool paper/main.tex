\documentclass{llncs}
\usepackage{amsmath,amssymb,amsfonts,latexsym,stmaryrd}
\usepackage[ruled,noend,nofillcomment,linesnumbered]{algorithm2e}
\usepackage{pslatex}
\usepackage[english]{babel}
\selectlanguage{english}

\usepackage{float}
\usepackage{pdfpages}
\usepackage{amssymb}
\usepackage[colorlinks]{hyperref}
\usepackage{cite}
\usepackage[utf8x]{inputenc}
\usepackage{textcomp}
\usepackage{ucs}
\usepackage{extarrows}
\usepackage{verbatim}
\usepackage{fancyvrb}
\usepackage[mathscr]{euscript}
\usepackage{color}
\usepackage{pifont}
\usepackage{subfigure,multimedia}
\usepackage{delarray}
\usepackage{graphicx}
\usepackage{tikz}
\usepackage{listings}
\usepackage{amsmath}
\usepackage{xspace}
\usepackage{cleveref}
\usepackage{wrapfig}
\usepackage{float}
\usepackage{pgf}
\usepackage{proof}
\usepackage{url}
\usetikzlibrary{chains,positioning,fit,shapes,arrows,calc,decorations.pathreplacing,decorations.pathmorphing}
\usetikzlibrary{shapes}
\usetikzlibrary{backgrounds,calc}

\newcommand{\MPass}{\textsc{MPass}}
\newcommand{\binary}{MPass}
\newcommand{\transition}{t}
\newcommand{\occvar}{{\tt occ}}
\newcommand{\occvarof}[1]{\occvar\left({#1}\right)}
\newcommand{\indexvar}{{\tt index}}
\newcommand{\indexvarof}[1]{\indexvar\left({#1}\right)}
\newcommand{\matchingvar}{{\tt match}}
\newcommand{\matchingvarof}[1]{\matchingvar\left({#1}\right)}

\begin{document}
\title{\MPass:  An Efficient  Tool for the Analysis of Message-Passing Programs\thanks{This research was in part funded by the
    Uppsala Programming for Multicore Architectures Research Center
    (UPMARC)}}

\author{Parosh Aziz Abdulla\inst{1} \and Mohamed Faouzi Atig\inst{1} \and Gaurav Saini\inst{2} \and Subham Modi\inst{3}}
\institute{Uppsala University, Sweden \and Indian Institute of Technology, Ropar \and Indian Institute of Technology, Kanpur}

\maketitle

\input abstract

\input introduction 

\input MPass-tool 

\input implementation

\input experimental

\newpage
\phantomsection
\addcontentsline{toc}{section}{References}
\bibliographystyle{abbrv}
\bibliography{biblio}

\input installation

\end{document}
